% 使用ctexart文档类(用XeLaTeX编译,直接支持中文)
\documentclass{ctexart}

%导言区,可以在此引入必要的宏包
\usepackage{tikz-flowchart}

% 各参数默认值
\flowchartset{
  free color = green, % 自由连线颜色(默认取green)
  norm color = blue, % 常规连线颜色(默认取blue)
  cong color = red, % 关联连线颜色(默认取red)
  proc fill color = white, % 顺序处理框填充颜色(默认取白色)
  test fill color = white, % 判断框填充颜色(默认取白色)
  io fill color = white, % 输入/输出框填充颜色(默认取白色)
  term fill color = white, % 开始/结束框填充颜色(默认取白色)
  chain direction = below, % 结点自动布置方向(默认取below)
  minimum node distance = 6mm, % 最小结点间距(默认取6mm)
  maximum node distance = 60mm, % 最大结点间距(默认取60mm)
  border line width = \pgflinewidth, % 各类流程框边框宽度(默认取当前线条宽度)
  flow line width = \pgflinewidth, % 各类流程线线条宽度(默认取当前线条宽度)
  stealth length = 1.5mm, % 箭头长度(默认取1.5mm)
  stealth width = 1.0mm, % 箭头宽度(默认取1.0mm)
}


\begin{document} %在document环境中撰写文档
% 可以绘制复杂流程图
\begin{tikzpicture}
  % -------------------------------------------------
  % 布置结点单元
  \node [proc, densely dotted, it] (p0) {新触发器消息线程};
  % 使用"join"参数以将新结点自动连接到前一个结点
  \node [term, join]      {触发器调度表};
  \node [proc, join] (p1) {获取配额 $k > 1$};
  \node [proc, join]      {打开队列};
  \node [proc, join]      {派发消息};
  \node [test, join] (t1) {获得消息?};
  % test结点的出口连接相对复杂,不再使用"join"参数
  % 继续布置完成所有结点
  \node [proc] (p2) {$k \mathbin{{-}{=}} 1$};
  \node [proc, join] (p3) {派发消息};
  \node [test, join] (t2) {获得消息?};
  \node [test] (t3) {容量满?};
  \node [test] (t4) {$k \mathbin{{-}{=}} 1$};
  % 将下一个结点布置到第2栏,直接布置于p1结点的右边
  % 由于栏间距已进行设置,在此无需再次设定
  \node [proc, fill=\freecolor!25, right=of p1] (p4) {重置拥塞};
  \node [proc, join=by free] {设置等待标志\textsc{mq}};
  \node [proc, join=by free] (p5) {派发消息};
  \node [test, join=by free] (t5) {获得消息?};
  \node [test] (t6) {容量满?};
  % 布置更多结点(应该避免这样布置结点,不是很好看)
  \node [test] (t7) [right=of t2] {$k \mathbin{{-}{=}} 1$};
  \node [proc, fill=\congcolor!25, right=of t3] (p8) {设置拥塞};
  \node [proc, join=by cong, right=of t4] (p9) {关闭队列};
  \node [term, join] (p10) {退出触发器消息线程};
  % -------------------------------------------------
  % 布置坐标点,以实现拐角连接或添加注释说明
  % 同时,用\cmark命令对各个点进行了标记,以方便调试
  \node [coord, right=0.8 of t1] (c1)  {}; \cmark{1}   
  \node [coord, right=0.8 of t3] (c3)  {}; \cmark{3}   
  \node [coord, right=0.8 of t6] (c6)  {}; \cmark{6}   
  \node [coord, right=0.8 of t7] (c7)  {}; \cmark{7}   
  \node [coord, left=0.8 of t4]  (c4)  {}; \cmark{4}   
  \node [coord, right=0.8 of t4] (c4r) {}; \cmark[r]{4}
  \node [coord, left=0.8 of t7]  (c5)  {}; \cmark{5}   
  % -------------------------------------------------
  % 布置注释结点
  \node [above=0mm of p4, it] {(队列为空)};
  \node [above=0mm of p8, it] {(队列非空)};
  % -------------------------------------------------
  % 首先绘制判断框与其它结点单元南北方向的连接线
  % 每次绘制时,先绘制一个带有一个固定位置标注的路径(path)
  % 然后再绘制箭头本身(arrow)。
  \path (t1.south) to node [near start, xshift=1em] {$y$} (p2);
  \draw [norm] (t1.south) -- (p2);
  \path (t2.south) to node [near start, xshift=1em] {$y$} (t3); 
  \draw [norm] (t2.south) -- (t3);
  \path (t3.south) to node [near start, xshift=1em] {$y$} (t4); 
  \draw [norm] (t3.south) -- (t4);
  \path (t5.south) to node [near start, xshift=1em] {$y$} (t6); 
  \draw [free] (t5.south) -- (t6);
  \path (t6.south) to node [near start, xshift=1em] {$y$} (t7); 
  \draw [free] (t6.south) -- (t7); 
  % ------------------------------------------------- 
  % 其次绘制判断框与其它结点单元东西方向的连接线
  % 为固定标位置,在垂直方向上进行坐标设定,然后再绘制箭头本身(arrow)。
  \path (t3.east) to node [near start, yshift=1em] {$n$} (c3); 
  \draw [cong] (t3.east) -- (p8);
  \path (t4.east) to node [yshift=-1em] {$k \leq 0$} (c4r); 
  \draw [norm] (t4.east) -- (p9);
  % -------------------------------------------------
  % 最后绘制其它方向的连接线
  % 同样先绘制一个带有一个固定位置标注的路径(path),然后再绘制箭头本身(arrow)。
  \path (t1.east) to node [near start, yshift=1em] {$n$} (c1); 
  \draw [free] (t1.east) -- (c1) |- (p4);
  \path (t2.east) -| node [very near start, yshift=1em] {$n$} (c1); 
  \draw [free] (t2.east) -| (c1);
  \path (t4.west) to node [yshift=-1em] {$k>0$} (c4); 
  \draw [norm] (t4.west) -- (c4) |- (p3);
  \path (t5.east) -| node [very near start, yshift=1em] {$n$} (c6); 
  \draw [free] (t5.east) -| (c6); 
  \path (t6.east) to node [near start, yshift=1em] {$n$} (c6); 
  \draw [free] (t6.east) -| (c7); 
  \path (t7.east) to node [yshift=-1em] {$k \leq 0$} (c7); 
  \draw [free] (t7.east) -- (c7)  |- (p9);
  \path (t7.west) to node [yshift=-1em] {$k>0$} (c5); 
  \draw [free] (t7.west) -- (c5) |- (p5);
  % -------------------------------------------------
  % 最后一个标不符合以上所有规则
  \draw [->, dotted, thick, shorten >=1mm]
  (p9.south) -- ++(5mm,-3mm)  -- ++(27mm,0) 
  |- node [black, near end, yshift=0.75em, it]
  {(当消息和资源都有效时)} (p0);
  % -------------------------------------------------     
\end{tikzpicture}

% 可以在结点中使用\verb等命令
\begin{tikzpicture}
  % 布置结点单元  
  \node [term] (st) {开始};
  \node [proc, join] (p1) {\verb!int r = 0!};
  \node [test, join] (t1) {\verb!m < n!};        
  \node [proc] (p2) {\verb! m ^= n !\\ \verb!n ^= m !\\ \verb!m ^= n!};        
  \node [test, join] (t2) {\verb|n != 0|};
  \node [proc] (p3) {\verb!r = m % n! \\ \verb!m = n!\\ \verb!n = r!};        
  \node [proc] (p4) {\verb!return m!};
  \node [term, join] (end) {结束};

  % 布置用于连接的坐标结点,同时为其布置调试标记点。
  \node [coord] (c1) at ($(p2.south)!0.5!(t2.north)$)  {}; \cmark{1}
  \node [coord, below = 0.25 of p3] (c2) {};  \cmark{2}
  \node [coord, above = 0.25 of p4] (c3) {};  \cmark{3}
  \node [coord, right = 1.0 of t1] (c4) {};  \cmark{4}
  \node [coord, right = 1.0 of t2] (c5) {};  \cmark{5}
  \node [coord, left = 1.0 of t2] (ct) {};  \cmark{t}        
  \node [coord] (c6) at (ct |- c2)  {}; \cmark{6}
       
  % 判断框连线,每次绘制时,先绘制一个带有一个固定
  % 位置标注的路径(path),然后再绘制箭头本身(arrow)。
  \path (t1.south) -- node [near start, right] {$Y$} (p2.north);
  \draw [norm] (t1.south) -- (p2.north);
  \path (t1.east) -- node [near start, above] {$N$} (c4);
  \draw [norm] (t1.east) -- (c4) |- (c1);
  \path (t2.south) -- node [near start, right] {$Y$} (p3.north);
  \draw [norm] (t2.south) -- (p3.north);
  \path (t2.east) -- node [near start, above] {$N$} (c5);
  \draw [norm] (t2.east) -- (c5) |- (c3) -- (p4.north);

  % 其它连线
  \draw [norm](p3.south) |- (c6) |- (c1);
\end{tikzpicture}

% 可以更改字号、各类流程框的宽度等
\begin{tikzpicture}[font=\small]
  % 布置结点单元
  \node [term] (st) {开始};
  \node [proc, join] (p1) {\verb|sum = x|\\\verb|x_pow = x|\\\verb|item = 0.0|};
  \node [proc, join] (p2) {\verb|n = 1|\\\verb|fact = 1|\\\verb|sign = 1|};        
  \node [proc, join, text width = 16em] (p3) {\verb|fact = fact * (n + 1) * (n + 2)|\\\verb|x_pow *= x * x|\\\verb|sign = -sign|\\\verb|item =  sign * x_pow / fact|\\\verb|sum += item|\\\verb|n += 2|};
        
  \node [test, join, text width = 10em] (t1) {\verb|fabs(item) > epsilon|};
  \node [proc] (p4) {\verb|return sum|};
  \node [term, join] (end) {结束};

  % 布置用于连接的坐标结点,同时为其布置调试标记点。
  \node [coord] (c1) at ($(p2.south)!0.5!(p3.north)$)  {}; \cmark{1}
  \node [coord, right = 2.0 of t1] (ct) {};  \cmark{t}        
  \node [coord] (c2) at (ct |- c1)  {}; \cmark{2}

  % 判断框连线,每次绘制时,先绘制一个带有一个固定
  % 位置标注的路径(path),然后再绘制箭头本身(arrow)。
  \path (t1.south) -- node [near start, right] {$N$} (p4.north);
  \draw [norm] (t1.south) -- (p4.north);
  \path (t1.east) -- node [near start, above] {$Y$} (ct);
  \draw [norm] (t1.east) -| (c2) -- (c1);  
\end{tikzpicture}

% 可以局部更改各参数
\flowchartset{
  proc fill color = orange!10, % 顺序处理框填充颜色(默认取白色)
  test fill color = green!30, % 判断框填充颜色(默认取白色)
  io fill color = blue!30, % 输入/输出框填充颜色(默认取白色)
  term fill color = red!30, % 开始/结束框填充颜色(默认取白色)
}

\begin{tikzpicture}
  % 布置结点单元
  \node [term] (st) {开始};
  \node [proc, text width = 6em, join] (p1) {\verb|int divisor|};       
  \node [test, join] (t1) {\verb|n <= 1|};
  \node [proc, text width = 6em] (p2) {\verb|divisor = 2|};
  \node [test, text width = 10em, join] (t2) {\verb|divisor * divisor <= n|};
  \node [test, text width = 8em] (t3) {\verb|n % divisor == 0|};
  \node [proc, text width = 6em] (p3) {\verb|divisor++|};
  \node [term, below = 1.6 of p3] (end) {结束};
  \node [proc, text width = 6em, left = 4.8 of t2] (p4) {\verb|return 0|};
  \node [proc, text width = 6em, right = 3.5 of p3] (p5) {\verb|return 0|};
  \node [proc, text width = 6em, right = 5.8 of t3] (p6) {\verb|return 1|};

  % 布置用于连接的坐标结点,同时为其布置调试标记点。
  \node [coord] (c1) at ($(p2.south)!0.5!(t2.north)$)  {}; \cmark{1}
  \node [coord, below = 0.25 of p3] (c2)  {}; \cmark{2}
  \node [coord, above = 0.5 of end] (c3) {};  \cmark{3}
  \node [coord, left = 0.5 of t2] (ct) {};  \cmark{t}
  \node [coord] (c4) at (c3 -| p5)  {}; \cmark{4}
  \node [coord] (c5) at (c2 -| ct)  {}; \cmark{5}
        
  % 判断框连线,每次绘制时,先绘制一个带有一个固定
  % 位置标注的路径(path),然后再绘制箭头本身(arrow)。
  \path (t1.south) -- node [near start, right] {$N$} (p2.north);
  \draw [norm] (t1.south) -- (p2.north);
  \path (t1.west) -| node [near start, above] {$Y$} (p4.north);
  \draw [norm] (t1.west) -| (p4.north);
        
  \path (t2.south) -- node [near start, right] {$Y$} (t3.north);
  \draw [norm] (t2.south) -- (t3.north);
  \path (t2.east) -| node [near start, above] {$N$} (p6.north);
  \draw [norm] (t2.east) -| (p6.north);
        
  \path (t3.south) -- node [near start, right] {$N$} (p3.north);
  \draw [norm] (t3.south) -- (p3.north);
  \path (t3.east) -| node [near start, above] {$Y$} (p5.north);
  \draw [norm] (t3.east) -| (p5.north);

  % 其它连线
  \draw [norm](p3.south) |- (c5) |- (c1);
  \draw [norm](p4.south) |- (c3);
  \draw [norm](p4.south) |- (c3) -- (end);
  \draw [norm](p5.south) -- (c4);
  \draw [norm](p6.south) |- (c3);
  \draw [norm](p6.south) |- (c3) -- (end);
\end{tikzpicture}

\end{document}

%%% Local Variables:
%%% mode: latex
%%% TeX-master: t
%%% End:
