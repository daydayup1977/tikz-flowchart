% 使用ctexart文档类(用XeLaTeX编译,直接支持中文)
\documentclass{ctexart}

%导言区,可以在此引入必要的宏包
\usepackage[debug]{tikz-flowchart}

% \flowchartset{
%   free color = gray,
%   norm color = blue,
%   cong color = red,
% }

\begin{document} %在document环境中撰写文档

\begin{tikzpicture}
  % -------------------------------------------------
  % 布置结点单元
  \node [proc, densely dotted, it] (p0) {新触发器消息线程};
  % 使用"join"参数以将新结点自动连接到前一个结点
  \node [term, join]      {触发器调度表};
  \node [proc, join] (p1) {获取配额 $k > 1$};
  \node [proc, join]      {打开队列};
  \node [proc, join]      {派发消息};
  \node [test, join] (t1) {获得消息?};
  % test结点的出口连接相对复杂,不再使用"join"参数
  % 继续布置完成所有结点
  \node [proc] (p2) {$k \mathbin{{-}{=}} 1$};
  \node [proc, join] (p3) {派发消息};
  \node [test, join] (t2) {获得消息?};
  \node [test] (t3) {容量满?};
  \node [test] (t4) {$k \mathbin{{-}{=}} 1$};
  % 将下一个结点布置到第2栏,直接布置于p1结点的右边
  % 由于栏间距已进行设置,在此无需再次设定
  \node [proc, fill=\freecolor!25, right=of p1] (p4) {重置拥塞};
  \node [proc, join=by free] {设置等待标志\textsc{mq}};
  \node [proc, join=by free] (p5) {派发消息};
  \node [test, join=by free] (t5) {获得消息?};
  \node [test] (t6) {容量满?};
  % 布置更多结点(应该避免这样布置结点,不是很好看)
  \node [test] (t7) [right=of t2] {$k \mathbin{{-}{=}} 1$};
  \node [proc, fill=\congcolor!25, right=of t3] (p8) {设置拥塞};
  \node [proc, join=by cong, right=of t4] (p9) {关闭队列};
  \node [term, join] (p10) {退出触发器消息线程};
  % -------------------------------------------------
  % 布置坐标点,以实现拐角连接或添加注释说明
  % 同时,用\cmark命令对各个点进行了标记,以方便调试
  \node [coord, right=of t1] (c1)  {}; \cmark{1}   
  \node [coord, right=of t3] (c3)  {}; \cmark{3}   
  \node [coord, right=of t6] (c6)  {}; \cmark{6}   
  \node [coord, right=of t7] (c7)  {}; \cmark{7}   
  \node [coord, left=of t4]  (c4)  {}; \cmark{4}   
  \node [coord, right=of t4] (c4r) {}; \cmark[r]{4}
  \node [coord, left=of t7]  (c5)  {}; \cmark{5}   
  % -------------------------------------------------
  % 布置注释结点
  \node [above=0mm of p4, it] {(队列为空)};
  \node [above=0mm of p8, it] {(队列非空)};
  % -------------------------------------------------
  % 首先绘制判断框与其它结点单元南北方向的连接线
  % 每次绘制时,先绘制一个带有一个固定位置标注的路径(path)
  % 然后再绘制箭头本身(arrow)。
  \path (t1.south) to node [near start, xshift=1em] {$y$} (p2);
  \draw [norm] (t1.south) -- (p2);
  \path (t2.south) to node [near start, xshift=1em] {$y$} (t3); 
  \draw [norm] (t2.south) -- (t3);
  \path (t3.south) to node [near start, xshift=1em] {$y$} (t4); 
  \draw [norm] (t3.south) -- (t4);
  \path (t5.south) to node [near start, xshift=1em] {$y$} (t6); 
  \draw [free] (t5.south) -- (t6);
  \path (t6.south) to node [near start, xshift=1em] {$y$} (t7); 
  \draw [free] (t6.south) -- (t7); 
  % ------------------------------------------------- 
  % 其次绘制判断框与其它结点单元东西方向的连接线
  % 为固定标位置,在垂直方向上进行坐标设定,然后再绘制箭头本身(arrow)。
  \path (t3.east) to node [near start, yshift=1em] {$n$} (c3); 
  \draw [cong] (t3.east) -- (p8);
  \path (t4.east) to node [yshift=-1em] {$k \leq 0$} (c4r); 
  \draw [norm] (t4.east) -- (p9);
  % -------------------------------------------------
  % 最后绘制其它方向的连接线
  % 同样先绘制一个带有一个固定位置标注的路径(path),然后再绘制箭头本身(arrow)。
  \path (t1.east) to node [near start, yshift=1em] {$n$} (c1); 
  \draw [free] (t1.east) -- (c1) |- (p4);
  \path (t2.east) -| node [very near start, yshift=1em] {$n$} (c1); 
  \draw [free] (t2.east) -| (c1);
  \path (t4.west) to node [yshift=-1em] {$k>0$} (c4); 
  \draw [norm] (t4.west) -- (c4) |- (p3);
  \path (t5.east) -| node [very near start, yshift=1em] {$n$} (c6); 
  \draw [free] (t5.east) -| (c6); 
  \path (t6.east) to node [near start, yshift=1em] {$n$} (c6); 
  \draw [free] (t6.east) -| (c7); 
  \path (t7.east) to node [yshift=-1em] {$k \leq 0$} (c7); 
  \draw [free] (t7.east) -- (c7)  |- (p9);
  \path (t7.west) to node [yshift=-1em] {$k>0$} (c5); 
  \draw [free] (t7.west) -- (c5) |- (p5);
  % -------------------------------------------------
  % 最后一个标不符合以上所有规则
  \draw [->, dotted, thick, shorten >=1mm]
  (p9.south) -- ++(5mm,-3mm)  -- ++(27mm,0) 
  |- node [black, near end, yshift=0.75em, it]
  {(当消息和资源都有效时)} (p0);
  % -------------------------------------------------     
\end{tikzpicture}
  
\end{document}

%%% Local Variables:
%%% mode: latex
%%% TeX-master: t
%%% End:
